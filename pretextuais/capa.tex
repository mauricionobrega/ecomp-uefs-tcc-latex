%%%%%%%%%%%%%%%%%%%%%%%%%%%%%%%%%%%%%%%%%%%%%%%%%%%%%%%%%%%%%%%%%%%%%%%%%%%%%%%
% Este arquivo faz parte do template de Relatório Parcial baseado nas normas da ABNT 
%  voltado para alunos da UEFS
% Desenvolvimento: Danilo de Oliveira Gonçalves
% Adaptação final: João Carlos Nunes Bittencourt
% Data: 31/03/2011
% Atualização: 30/11/2011
% Descrição do arquivo:
%   Esse arquivo apresenta as definições de constantes que formarão a capa e 
%   a folha de rosto. Siga as instruções e modifique de acordo com o que
%   lhe foi orientado.
%%%%%%%%%%%%%%%%%%%%%%%%%%%%%%%%%%%%%%%%%%%%%%%%%%%%%%%%%%%%%%%%%%%%%%%%%%%%%%%

% ---------- Preambulo ----------
\instituicao{Universidade Estadual de Feira de Santana} % nome da instituicao
\departamento{Colegiado do curso de Engenharia de Computação}
\graduacao{Bacharelado em Engenharia de Computação} % nome do curso
\curso{Engenharia de Computação}

\documento{Trabalho de Conclusão de Curso} % tipo de documento
\titulacao{Bacharel} % [Bacharel]

\titulo{Título em Português} % titulo do trabalho em portugues
\subtitulo{Sub-título, se necessário} % caso necessário um sub-título, utilize este campo
\title{Title in English} % titulo do trabalho em ingles

\autor{Nome Completo} % autor do trabalho
\cita{SOBRENOME, Nome} % sobrenome (maiusculas), nome do autor do trabalho

\palavraschave{Palavra-chave 1. Palavra-chave 2. ...} % palavras-chave do trabalho, separados por ponto
\keywords{Keyword 1. Keyword 2. ...} % palavras-chave do trabalho em ingles, separados por ponto

\comentario{\UEFSdocumentodata\ apresentado ao \UEFSdepartamentodata\ como requisito parcial para obtenção do grau de \UEFStitulacaodata\ em \UEFScursodata\ pela \ABNTinstituicaodata.}

\orientador{Nome do Orientador} % nome do orientador do trabalho
%\orientador[Orientadora:]{Nome da Orientadora} % <- no caso de orientadora, usar esta sintaxe
\coorientador{Nome do Co-orientador} % nome do co-orientador do trabalho, caso exista
%\coorientador[Co-orientadora:]{Nome da Co-orientadora} % <- no caso de co-orientadora, usar esta sintaxe

\local{Feira de Santana} % cidade
\data{2011} % ano



